
\documentclass[sigplan,screen]{acmart}

%
% defining the \BibTeX command - from Oren Patashnik's original BibTeX documentation.
\def\BibTeX{{\rm B\kern-.05em{\sc i\kern-.025em b}\kern-.08emT\kern-.1667em\lower.7ex\hbox{E}\kern-.125emX}}
    

\copyrightyear{2018}
\acmYear{2018}
\setcopyright{acmlicensed}

%
% These commands are for a JOURNAL article.
%\setcopyright{acmcopyright}
%\acmJournal{TOG}
%\acmYear{2018}\acmVolume{37}\acmNumber{4}\acmArticle{111}\acmMonth{8}
%\acmDOI{10.1145/1122445.1122456}


\begin{document}

%
% The "title" command has an optional parameter, allowing the author to define a "short title" to be used in page headers.
\title{Design, Architecture, and Communication in Agile Development}

%
% The "author" command and its associated commands are used to define the authors and their affiliations.
% Of note is the shared affiliation of the first two authors, and the "authornote" and "authornotemark" commandsonline
% used to denote shared contribution to the research.
\author{David Lahtinen}
\authornote{Both authors contributed equally to this research.}
\email{dlahtine@asu.edu}
\author{Ruihao Zhou}
\authornotemark[1]
\email{rzhou21@asu.edu}

%
% By default, the full list of authors will be used in the page headers. Often, this list is too long, and will overlap
% other information printed in the page headers. This command allows the author to define a more concise list
% of authors' names for this purpose.
\renewcommand{\shortauthors}{Trovato and Tobin, et al.}

%
% The abstract is a short summary of the work to be presented in the article.
\begin{abstract}

\end{abstract}

%
% The code below is generated by the tool at http://dl.acm.org/ccs.cfm.
% Please copy and paste the code instead of the example below.
%
\begin{CCSXML}
<ccs2012>
  <concept>
    <concept_id>10011007.10011074.10011134.10011135</concept_id>
    <concept_desc>Software and its engineering~Programming teams</concept_desc>
    <concept_significance>500</concept_significance>
  </concept>
</ccs2012>
\end{CCSXML}

\ccsdesc[500]{Software and its engineering~Programming teams}

%
% Keywords. The author(s) should pick words that accurately describe the work being
% presented. Separate the keywords with commas.
\keywords{Agile}

%
% This command processes the author and affiliation and title information and builds
% the first part of the formatted document.
\maketitle

\section{Introduction}
Our SER 574 instructor tasked us with researching communications in Agile teams, particularly with respect to architecture and design. This is a well established field, and we won't be making any novel observations, so we will not have a "results" section.

\section{Important things within Agile Teams}
How do different Agile teams deliver high quality product when lots of problems are ahead of them? Each team has their own "implementation" of Agile and how they dismiss the big difference across them? By taking a research about this top, except for some core principles each team would adhere, there are several important things should be considered to collaboration and communication between Agile teams.

\subsection{Goal}
Each team should have clear idea about the final objectives and separation of work. If everyone has his own ideas about project, it's going to be really hard to continue the project. They should pay their efforts at same goal and work for it. Multiple teams work at same time will make some useless working and it requires each team to create suited independence to minimize this. Therefore, product backlog is popular among developers which can help them identify and plan for cross-team dependencies.

\subsection{Standards} 
Each team should agree on the standards they would adopt and follow before the project starts. For example, which kind of development procedure will each team use and how they will create testing for each phase. Of course, standards also includes patterns, code style and more. The whole work efforts across means it's a huge product and it's appropriate to create checking points in order to check errors more easily and release developer's burden. Also, all team should have full access to the whole project repository and contribute at anytime.

\subsection{Communication}
Each team should have communication quickly, frequently, quickly as soon as possible. It's all about teamwork. Using some project management tools such as Slack, Git and more to convey information. Daily and face to face meeting are two core ideas in Agile development. If you have ideas, say it before all the team or summarize it as a demo to let every team and its member know. Before taking actions, you should get their agreement. It's not mandatory required every one should attend, but each team should have one representative to show other team leaders that what's the trouble you are facing and what's you next plan and so on. "Communication improves, teamwork improves and "Us vs them" is reduced. And if the distance is not very long, seed visits infrequent intervals across the length of the project, it builds trust between the client and the team and between the team members themselves. Don't forget, that the highest performing teams are those based on trust!" [1]

Feedback, each team should incorporate feedback quickly form other teams and come up with solutions to fix it before it's going to deeper development. Take Scrum for example, work is divided into small iterations called Sprint which includes a lists of tasks. Since each Sprint's size is not big, they can work on it with expected time. Once finish one, they can easily reflect whether it's good or not and make some necessary change. And when one Sprint is finished, each team is able to evaluate it and decide which parts should be kept or altered. Changes can be incorporated  into next iteration which can save lots of time and increase productivity.

\section{Communication across Agile Teams}
As Dr. Mehlhase spake to us in class, Architecture is design at the highest level. At this level, the Agile teams cannot work separately from the rest; there must be communication and coordination. As the government describes in its article about Agile team communication, they recommend holding regular -daily if possible- scrum meetings consisting of representatives from related agile teams so that each team is on board with the bigger picture. \cite{Lulit} If your project has multiple project owners, they should also meet regularly so that releases and features can be planned, but also so that each team can learn from the mistakes and novel solutions of the others. \cite{Lulit} The closer to the modular level that you examine, however, the less communication to other teams about it is required, as they likely only want to know the interfaces.
\bibliographystyle{ACM-Reference-format}
\bibliography{refs}
\end{document}
